\documentclass[t]{beamer}
\usetheme{Warsaw}
\usepackage{array}
%\usepackage{graphicx}
\usepackage{amssymb,amsmath,mathrsfs,amsfonts}
%\usepackage[colorhighlight,display]{texpower}
%\usepackage{caption}
%\usepackage[all]{xy}
\usepackage{beamerthemesplit}
\mode<presentation>
%\usepackage{pause}
\usepackage{ulem}  % for strikethroughs
\usepackage{cancel} % for strikethroughs in math mode 
\usepackage{tikz}
\usetikzlibrary{shapes}
\usepackage{hyperref}
\hypersetup{pdfpagemode=FullScreen}
\usepackage{ifthen}
\usepackage{animate}
\usepackage{color}
\usepackage{type1cm}  % used for watermarking
\usepackage{eso-pic}  % used for watermarking


\theoremstyle{plain}
\newtheorem{prop}{Proposition}
\newtheorem{thm}[prop]{Theorem}
\newtheorem{lem}[prop]{Lemma}
\newtheorem{cor}[prop]{Corollary}
\theoremstyle{definition}
\newtheorem{dfn}{Definition}
\newtheorem{rem}[prop]{Remark}
\newtheorem{ex}{Example}[section]
%\newtheorem{note}{Note}[section]
\newtheorem{exercise}{Exercise}[section]
\newcommand{\nin}{\noindent}
\newcommand{\ds}{\displaystyle}
\renewcommand{\figurename}{Figure \arabic{figure}}



\renewcommand*\familydefault{\sfdefault} 




%%%%%%%%%%%%%%%%%%%%%%%%%%5
%%%%%%%%%%%%%%%%%%%%%%%%%%%%
%%%% some commands that have different meaning in the article/presentation modes

\newcommand{\vvfill}{\mode<presentation>{\vfill}  \mode<article>{\medskip}}   %vfill in presentation only
\newcommand{\sketchspace}{ 
\mode<article>{ \medskip\noindent{\textbf{Sketch:}} \vspace*{6cm} }
\mode<presentation>{ } 
}
\newcommand{\examplespace}{ 
\mode<article>{ \medskip\noindent{\textbf{Example:}} \vspace{6cm} }
\mode<presentation>{ } 
}
\newcommand{\artsmspace}{\mode<article>{\vspace*{2cm}} }  %small space in article mode
\newcommand{\artlargespace}{\mode<article>{\vspace*{6cm}} }  %large space in article mode

\newcommand{\dx}{\,dx}

\newcommand{\soln}{{\textbf{Solution: }}\,\,\,}


\newcommand{\makedate}{\vvfill
\begin{picture}(10,10)  
\put(260,-20){\mbox{\tiny{\today}}}
\end{picture}
}

\newcommand{\pd}[2]{\dfrac{\partial#1}{\partial#2}}
\newcommand{\pD}[2]{\dfrac{\partial^2#1}{\partial#2^2}}
\newcommand{\pdd}[3]{\dfrac{\partial^2#1}{\partial#2 \partial#3}}


\normalem %stops the ulem package making all the emphs into underlines....
 
 
 
 \newcommand{\refandrev}[2]{
 \begin{small}
  \hspace{6cm}
  \begin{minipage}[r]{8cm}
  Stewart,    Chapter #1   \\
  Review:  \parbox[t]{6cm}{#2}
\end{minipage}
\end{small}
}



\newcounter{heading}
\setcounter{section}{1}
\setcounter{heading}{0}

\newcommand{\makeheading}[1]{\medskip\begin{large}\noindent\textbf{{#1}}\end{large}\smallskip}

%\newenvironment{head}[1]{\medskip\stepcounter{heading}\noindent\textbf{\hspace{0.2cm}{#1}.}}{}
\newcommand{\newhead}[1]{\medskip\stepcounter{heading}\noindent\textbf{\hspace{0.2cm}{#1}.}}


\newcommand{\pf}[1]{\noindent\textit{Proof.}\vspace*{#1 cm}}
\newcommand{\sol}[1]{\noindent\textit{Solution.}\vspace*{#1 cm}}
\newcommand{\further}[1]{\begin{small}\noindent\textit{Further reading: #1}\end{small}}
\newcommand{\exr}[1]{\begin{footnotesize}\noindent\textit{\textbf{Exercises:} Stewart #1}\end{footnotesize}}


% Sets of numbers
\newcommand{\C}{\mathbb{C}}
\newcommand{\RR}{\mathbb{R}}
\newcommand{\Z}{\mathbb{Z}}
\newcommand{\N}{\mathbb{N}}
\newcommand{\Q}{\mathbb{Q}}

% Partitions
\newcommand{\PP}{\mathcal{P}}

% Limits
\newcommand{\limm}[1]{\displaystyle \lim_{x\to #1}}

% Backslash
\newcommand{\bs}{\backslash}

% functions
\newcommand{\cosec}{\mathrm{cosec}}
\newcommand{\cosech}{\mathrm{cosech}}
\newcommand{\sech}{\mathrm{sech}}
\newcommand{\Li}{\mathrm{Li}}
\newcommand{\si}{\mathrm{Si}}
\newcommand{\erf}{\mathrm{erf}}

% Domain and Range
\newcommand{\Dom}{\mathrm{Dom}}
\newcommand{\Codom}{\mathrm{Codom}}
\newcommand{\Range}{\mathrm{Ran}}



\title{Week 1:   Functions}
\date{July 23 -- July 27, 2012}

\begin{document}

\frame{\titlepage}

\setcounter{tocdepth}{2}
\frame{\tableofcontents

\begin{flushright}
\hyperlink{tues}{\beamergotobutton{Lecture 2}}
\end{flushright} 
}

\AtBeginSection[]
{
\begin{frame}<beamer> 
\tableofcontents[currentsection]  % show TOC and highlight current section
\end{frame}
}


\frame
{
\frametitle{Who am I?}
\begin{itemize}
\item[Name] Anandam Banerjee
\item[Email] anandam.banerjee@anu.edu.au
\item[Office] 2140 JD
\item[Phone] (02) 6125 7701
\item[{\small Office hours}$\!\!\!$] Monday 3-30 to 4-30 pm,\\
Tuesday 2 to 3 pm,\\
Wednesday 2 to 3 pm.
%\item[URL] {\small\url{http://maths.anu.edu.au/~banerjee/Math1014notes.html}}
\end{itemize}}

\section{Introduction}
\subsection{Sets}
\begin{frame}
\makeheading{Sets}

A \emph{set} is a collection of well defined and distinct objects.  

\uncover<+->{
\newhead{Sets of Numbers}

\begin{itemize}
\item $\mathbb N$

\item $\Z$

\item $\Q$

\item $\RR$

\end{itemize}}

\uncover<+->{
\newhead{Set notation} If $A$ is a set of numbers and the number $x$ is a member of the set $A$, then we write $x\in A$.
If $x$ is not a member of $A$ then we write $x\notin A$.}

\uncover<+->{
\newhead{Example} True or false?
\[\frac{1}{2}\in\Z;\qquad\sqrt{2}\in\RR;\qquad3\in\N;\qquad\pi\notin\Q.\]}

\end{frame}

\begin{frame}
\newhead{Intervals} Suppose that $a, b \in \RR$ and $a<b$.\pause
\begin{itemize}[<+->]
\item $(a,b]=\{x\in\RR:a<x\leq b\}$
\vspace*{.3cm}

\noindent (The colon `:' is read as \textit{such that}.) \newline
$a$ and $b$ are called \textit{endpoints} of the interval. \newline
$b$ is included in the interval; $a$ is not.

\item $[a,b]=\{x\in\RR:a\leq x\leq b\}$
\vspace*{.2cm}
\item $(a,b)=\{x\in\RR:a<x<b\}$
\vspace*{.2cm}
\item Rays on the real line can be expressed as intervals.\newline
\vspace*{.2cm}

\item \textit{Note:} $\infty$ and $-\infty$ are \textit{not} real numbers.
\item $[a,b]$ is a \textit{closed} interval; \newline
$(a,b)$ is an \textit{open} interval; \newline
$(a,b]$ is neither open nor closed.
\end{itemize}


\end{frame}

\begin{frame}

\begin{dfn} Suppose that $A$ and $B$ are two sets. We say that $A$ is a \textit{subset} of $B$ if $x\in A$ implies that $x\in B$, and we will denote this by $A \subseteq B$. If $A$ is a subset of $B$ then we also say that $B$ \textit{contains} the set $A$.\end{dfn}\pause

\vspace*{1cm}

\newhead{Example} True or false?
\begin{itemize}[<+->]
\item $\{3,4,5,6\}$ is a subset of $\{2,3,4,5,6\}$
\item $\Z$ is a subset of $\N$
\item $\N$ is a subset of $\Z$
\item $(2,8]$ is a subset of $[2,8]$
\item $[2,8)$ is a subset of $(2,8]$
\item $[3, 4]$ is a subset of $\Q$
\item $\Q$ is a subset of $\RR$
\end{itemize}

\end{frame}

\subsection{Inequalities}
\begin{frame}
\makeheading{Inequalities}

\uncover<+->{\newhead{Solving Inequalities}}
\begin{itemize}[<+->]
\item \textit{When you multiply or divide by a negative quantity, reverse the inequality sign.}
\item Taking reciprocals of inequlities is tricky. It is usually better to multiply and divide instead.
\item To remove an unknown from the denominator, either (i) multiply by the \textit{square} of the denominator or (ii) break into cases.
\end{itemize}


\uncover<+->{\newhead{Examples}}
\begin{enumerate}[<+->]
\item Solve $x^2-x-6>0$.
\item Solve $\ds\frac{1}{3-x}<\frac{1}{5}$.
\item Solve $\ds x\geq\frac{3}{x-2}$.
\item Solve $\ds\frac{3}{1-x}<\frac{2}{x}$.
\end{enumerate}
\end{frame}

\begin{frame}
\newhead{Absolute values}

\begin{enumerate}[<+->]
\item \textit{Definition.} If $x\in\RR$ then $|x|$ is defined by
\[|x|=\begin{cases}
x&\mbox{if }x\geq0\\
-x&\mbox{if }x<0.
      \end{cases}\]
\item \textit{Properties.}
\begin{itemize}
\item $|-x|=|x|$
\item $|xy|=|x||y|$
\item $\displaystyle\left|\frac{x}{y}\right|=\frac{|x|}{|y|}$ if y is nonzero
\item The triangle inequality: $|x+y|\leq|x|+|y|$
\item $|x|=\sqrt{x^2}$ and $|x|^2=x^2$
\end{itemize}
\item \textit{Inequalities.}
\begin{itemize}
\item $|y|<a\qquad\mbox{iff}\qquad -a<y<a$
\item $|y|>a\qquad\mbox{iff}\qquad y<-a\quad\mbox{or}\quad y>a$
\end{itemize}
\item \textit{Geometric Interpretation.}
\end{enumerate}

\end{frame}

\begin{frame}
\newhead{Example} Solve\\
(a) $|x-4|<3$ \qquad\qquad\pause (b) $|3x+5|\geq 2$ \qquad\qquad\pause (c) $\ds\left|\frac{x-1}{x-5}\right|<1$.
\end{frame}

\section{Functions}
\begin{frame}
\makeheading{Functions}

\uncover<+->{\noindent Many naturally occurring quantities that vary with time can be modelled using \textit{functions}.\\}
\uncover<+->{\begin{ex} The volume of water stored in Lake Burley-Griffin is a variable that depends on time. For any particular time $t$, we could denote the corresponding volume by $f(t)$. In this situation}
\begin{itemize}[<+->]
\item $f$ is called a function,
\item if $t$ is an input for the function $f$, then $f(t)$ is the\\ corresponding output.
\end{itemize}
\end{ex}

\uncover<+->{\noindent In this course, we begin a mathematical study of functions whose inputs and outputs are real numbers.}


\end{frame}

\subsection{Representation of functions}
\begin{frame}\label{tues}\uncover<+->{\newhead{Specifying a function}

\noindent A function $f$ has \textit{two} parts to its definition:}
\begin{enumerate}[<+->]
\item the rule which explains how to get outputs from inputs,
\item the specification of the function's \textit{domain} (i.e. set of inputs).
\end{enumerate}
\uncover<+->{A key point is that any input must give \emph{exactly one} output.}

\uncover<+->{\newhead{Example and terminology} A function $f$ with domain $[0,\infty)$ is given by the rule
\[f(x)=x^2\qquad\forall x\in[0,\infty).\]}



\end{frame}

\begin{frame}
\uncover<+->{\newhead{The maximal domain} If the domain of a function is not specified, but the function rule is, then the default domain, known as the \textit{maximal} or \textit{natural domain}, is the largest possible domain for which the rule makes sense.}

\uncover<+->{\newhead{Example} True or false?}
\begin{itemize}[<+->]
\item The maximal domain of $\ds f(x)=x^2$ is $[0,\infty)$.
\item The maximal domain of $\ds f(x) = \frac{1}{x}$ is $\mathbb{R}$.
\item The maximal domain of $\ds f(x) = \sqrt{x}$ is $[0,\infty)$.

\end{itemize}

\end{frame}

\begin{frame}
\uncover<+->{\newhead{The range of a function} Suppose that $f$ is a function. The \textit{range} of $f$, denoted by $\Range(f)$, is defined by
\[\Range(f)=\{f(x)\in B:x\in\Dom(f)\}.\]}
\uncover<+->{\noindent Note that $\Range(f)$ is the set of all output values for $f$.  Note also that the range of $f$ depends on the domain of $f$.  If you are asked to give the range of a function $f$ where the domain is not specified, you may assume that the domain is maximal.}

\uncover<+->{\newhead{Example} Suppose $f(x) = x^{2}$.}
\begin{itemize}[<+->]
\item If the domain of $f$ is $\mathbb{R}$, the range of $f$ is $[0, \infty)$.
\item If the domain of $f$ is $[0,\infty)$, the range of $f$ is still $[0, \infty)$.
\item If the domain of $f$ is $[2, 4]$, the range of $f$ is $[4, 16]$.
\end{itemize}
\end{frame}

\begin{frame}
\uncover<+->{\newhead{The graph of a function} If $f$ is a function then its graph is the set of ordered pairs
\[\{(x,f(x)):x\in\Dom(f)\}.\]
This is just the set of all input-output pairs. If $y=f(x)$ the graph can be represented in the $xy$-plane. 
\vspace*{.3cm}}

\uncover<+->{\noindent Note that not all curves in the $xy$ plane are graphs of functions.  In fact, a curve in the $xy$ plane is the graph of a function if and only if no vertical line intersects the curve more than once (the \emph{vertical line test}).}

\uncover<+->{\vspace*{.3cm}
\noindent \emph{Note that although the concepts of even and odd functions and increasing and decreasing functions will not be covered in lecture, you are still expected to be familiar with them} (refer to the textbook).}
\vfill
\end{frame}

\subsection{A catalogue of functions}
\begin{frame}
\uncover<+->{\makeheading{Special classes of functions}

\noindent In this section we list some familar classes of functions.}

\uncover<+->{\newhead{Polynomials} A function $f:\RR\to\RR$ is called a \textit{polynomial} if
\[f(x)=a_0+a_1x+a_2x^2+\cdots+a_nx^n\]
where $a_0,a_1,a_2,\ldots,a_n\in\RR$ and $n\in\N$.  The $a_{j}$ are called the \emph{coefficients} of the polynomial and $n$ is the \emph{degree} of the polynomial (as long as $a_{n}$ is nonzero).  If $n=1$ the polynomial is called linear, $n=2$ quadratic, $n=3$ cubic, etc.

\vfill}


\uncover<+->{\newhead{Rational functions} Suppose that $p$ and $q$ are polynomials. A function $f$ is called a \textit{rational function} if
\[\Dom(f)=\{x\in\RR:q(x)\neq0\}\]
and
\[f(x)=\frac{p(x)}{q(x)}\qquad\forall x\in\Dom(f)\]}
\end{frame}

\begin{frame}
\uncover<+->{\newhead{Trigonometric functions} You should be familar with the functions $\sin$, $\cos$, $\tan$, $\sec$, $\cosec$ and $\cot$.

\smallskip}

\uncover<+->{\noindent\textit{Important note:} In calculus, it is essential that angles are given in radian measure. Recall that $360$ degrees is equal to $2\pi$ radians.

\vspace*{.3cm}
In particular, you should know (this list is not exhaustive!)}
\begin{itemize}[<+->]
\item That sine and cosine are periodic with a period of $2\pi$. 
\item That the range of sine and cosine is $[-1, 1]$.
\item That $sin(x) = 0$ if $x \in \{0, \pm \pi, \pm 2\pi, \pm 3\pi, \ldots\}$ and $cos(x) = 0$ if $x \in \{\pm \frac{\pi}{2}, \pm \frac{3\pi}{2}, \pm \frac{5\pi}{2}, \ldots\}$.
\item The values of $x$ for which $sin(x)$ equals $1$ or $-1$ and the values of $x$ for which $cos(x)$ equals $1$ or $-1$.
\item How to draw the graphs of sin, cos, tan (and to a lesser extent how to draw the graphs of sec, cosec, and cot).
\item How to compute sin, cos, tan, sec, cosec, and cot of the values $\frac{\pi}{6}, \frac{\pi}{4}, \frac{\pi}{3}, \frac{\pi}{2}$, etc.
\end{itemize} 
\end{frame}

\begin{frame}
\uncover<+->{\noindent\textit{Important} -- make sure you know the standard trig formulae:}
\begin{itemize}[<+->]
\item complementary identities
\begin{align*}
\sin\left(\frac{\pi}{2}-x\right)&=\cos x\\
\cos\left(\frac{\pi}{2}-x\right)&=\sin x
\end{align*}
\item Pythagorean identities
\begin{align*}
\cos^2x+\sin^2x&=1\\
1+\tan^2x&=\sec^2x\\
\cot^2x+1&=\cosec^2x
\end{align*}
\item the sum and difference formulae
\begin{align*}
\sin(x\pm y)&=\sin x\cos y\pm\cos x\sin y\\
\cos(x\pm y)&=\cos x\cos y\mp\sin x\sin y\\
\tan(x\pm y)&=\frac{\tan x\pm\tan y}{1\mp\tan x\tan y}
\end{align*}
\end{itemize} 
\end{frame}

\begin{frame}
\uncover<+->{\noindent\textit{Important} -- make sure you know the standard trig formulae:}
\begin{itemize}[<+->]
\item double-angle formulae
\begin{align*}
\sin(2x)&=2\sin x\cos x\\
\cos(2x)&=\cos^2x-\sin^2x\\
\tan(2x)&=\frac{2\tan x}{1-\tan^2x}.
\end{align*}
\end{itemize}

\end{frame}

\begin{frame}
\uncover<+->{\newhead{The exponential and logarithm functions}

\begin{dfn}
Functions given as $f(x)=a^x, a>0, a\in\RR$ are called exponential functions. $\Dom(f)=\RR,\, \Range(f)=(0,\infty)$.
\end{dfn}}
\vspace*{.5cm}

\uncover<+->{\begin{dfn}
Logarithmic functions, denoted $\log_a$, is the inverse of $a^x$ for $a>0, a\in\RR$:
\begin{itemize}
\item $\log_a(a^x)=x$, $a^{\log_ax}=x$.
\item $\Dom(\log_a)=(0,\infty)$ and $\Range(\log_a)=\RR$.
\end{itemize}
\end{dfn}}

\uncover<+->{\newhead{Roots}\\
Functions of the form $x^{\frac{1}{n}}$. e.g.\ $x^{\frac{1}{3}}=\sqrt[3]{x}$.
}
\end{frame}

\begin{frame}
\frametitle{Constructing new functions from known ones}
\uncover<+->{\newhead{Combining functions}If two functions $f$ and $g$ have the same domain $A$, we can construct new functions $f+g$, $f-g$ and $f\cdot g$ each with domain $A$. 
These are defined \textit{pointwise} by the following formulae:}
\uncover<+->{\begin{align*}
(f+g)(x)&=f(x)+g(x)\\
(f-g)(x)&=f(x)-g(x)\\
(f\cdot g)(x)&=f(x)g(x).\\
\intertext{\vspace*{-.2cm}We can also define $\dfrac{f}{g}$ by the formula}
\left(\frac{f}{g}\right)(x)&=\frac{f(x)}{g(x)}\qquad\qquad\mbox{provided that }g(x)\neq0.
\end{align*}}
\uncover<+->{The domain of $\ds\frac{f}{g}$ is $\{x\in A:g(x)\neq 0\}$.}
\end{frame}

\begin{frame}
\frametitle{Constructing new functions from known ones}
\uncover<+->{\noindent If $f$ has domain $A$ and $\Range(f)$ is a subset of $\Dom(g)$ then we can define a new function $g\circ f$ on $A$ by the rule
\[(g\circ f)(x)=g(f(x))\qquad\forall x\in A\]
The function $g\circ f$ is called the \textit{composite function} of $g$ and $f$.}

\uncover<+->{\begin{ex} \[\cos(x^2)=f\circ g(x),\] where $f(x)=\cos(x)$ and $g(x)=x^2$.
\end{ex}}
\end{frame}

\begin{frame}
\frametitle{Examples}
\begin{itemize}[<+->]
\item Let $f(x) = x^{2}$ with domain $[0, \infty)$ and $g(x) = sin(x)$ with domain $\mathbb{R}$.  Then $(g\circ f)(x) = sin(x^{2})$ and $(f\circ g)(x) = (sin(x))^{2}$.  Note the parentheses!  What are the domains and ranges of $g\circ f$ and $f \circ g$? 
\vspace*{.2cm}

\item Let $f(x) = cos(x)$ and $g(x) = e^{x}$, then $(g \circ f)(x) = e^{cos(x)}$.  Note that we didn't specify the domains, so we implicitly mean the maximal domain of $f$ such that $g \circ f$ makes sense.

\vspace*{.2cm}
\item Let $f(x) = x^3$ with domain $[-2, 2]$ and $g(x) = ln(x)$ with maximal domain $(0, \infty)$.  Does $g\circ f$ make sense on the domain of $f$?  Note that if we write an expression like $ln(x^{3})$ where we don't specify the domain of $x^{3}$, we implicitly mean that the domain of $ln(x^{3})$ is the maximal one.
\end{itemize}
\end{frame}

\begin{frame}
\frametitle{Piecewise defined functions}
\uncover<+->{\noindent A piecewise defined function is given by different formulas on different pieces of its domain.}

\uncover<+->{\newhead{Example}
\[f(x)=\begin{cases}
1-x&\mbox{if }x\leq 1\\
x^{2}&\mbox{if }x>1.
      \end{cases}\]}

\uncover<+->{\vspace*{1cm} \noindent Can you think of another piecewise defined function that we encountered in the first lecture?}
\end{frame}

\begin{frame}
\uncover<+->{\newhead{Function transformations} Once the shapes of the graphs of basic functions are known, we can alter the shape of the graph by altering the function in simple ways. }


\uncover<+->{\noindent Suppose that $a>0$ and $c>1$.}

\uncover<+->{\begin{center}
\begin{tabular}{| c | l |}
\hline
New graph & Obtained from $y=f(x)$ by \\
\hline
$y=f(x)+a$ & translating the graph upwards by $a$ units \\
$y=f(x)-a$ & translating the graph downwards by $a$ units \\
$y=f(x+a)$ & translating the graph to the left by $a$ units \\
$y=f(x-a)$ & translating the graph to the right by $a$ units \\
\hline
$y=cf(x)$ & stretching the graph vertically by factor $c$ \\
$y=(1/c)f(x)$ & compressing the graph vertically by factor $c$ \\
$y=f(cx)$ & compressing the graph horizontally by factor $c$ \\
$y=f(x/c)$ & stretching the graph horizontally by factor $c$ \\
\hline
$y=-f(x)$ & reflecting the graph about the $x$-axis \\
$y=f(-x)$ & reflecting the graph about the $y$-axis \\
\hline
\end{tabular}
\end{center}}
\end{frame}

\begin{frame}
\newhead{Examples} Sketch the graphs given by
\begin{enumerate}[<+->]
\item[(a)] $y=\sqrt{2-x}$
\item[(b)] $y=3\sin(2x+\pi)$
\end{enumerate}
\end{frame}

\end{document}