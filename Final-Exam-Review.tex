\documentclass[10pt]{amsart}
\usepackage{array}
\usepackage{graphicx}
\usepackage{amssymb,amsmath,mathrsfs,amsfonts}
%\usepackage[colorhighlight,display]{texpower}
%\usepackage{caption}
%\usepackage[all]{xy}
%\usepackage{pause}
\usepackage{ulem}  % for strikethroughs
\usepackage{cancel} % for strikethroughs in math mode 
\usepackage{tikz}
\usepackage{calc}
\usetikzlibrary{shapes}
\usepackage{hyperref}
%\hypersetup{pdfpagemode=FullScreen}


\theoremstyle{plain}
\newtheorem{prop}{Proposition}
\newtheorem{thm}[prop]{Theorem}
\newtheorem{lem}[prop]{Lemma}
\newtheorem{cor}[prop]{Corollary}
\theoremstyle{definition}
\newtheorem{dfn}{Definition}
\newtheorem{rem}[prop]{Remark}
\newtheorem{ex}{Example}[section]
%\newtheorem{note}{Note}[section]
\newtheorem{exercise}{Exercise}[section]
\newcommand{\nin}{\noindent}
\newcommand{\ds}{\displaystyle}
\renewcommand{\figurename}{Figure \arabic{figure}}



%\renewcommand*\familydefault{\sfdefault} 




%%%%%%%%%%%%%%%%%%%%%%%%%%5
%%%%%%%%%%%%%%%%%%%%%%%%%%%%
%%%% some commands that have different meaning in the article/presentation modes

\newcommand{\vvfill}{\mode<presentation>{\vfill}  \mode<article>{\medskip}}   %vfill in presentation only
\newcommand{\sketchspace}{ 
\mode<article>{ \medskip\noindent{\textbf{Sketch:}} \vspace*{6cm} }
\mode<presentation>{ } 
}
\newcommand{\examplespace}{ 
\mode<article>{ \medskip\noindent{\textbf{Example:}} \vspace{6cm} }
\mode<presentation>{ } 
}
\newcommand{\artsmspace}{\mode<article>{\vspace*{2cm}} }  %small space in article mode
\newcommand{\artlargespace}{\mode<article>{\vspace*{6cm}} }  %large space in article mode

\newcommand{\dx}{\,dx}

\newcommand{\soln}{{\textbf{Solution: }}\,\,\,}
\newcommand{\disp}{\displaystyle}

\newcommand{\makedate}{\vvfill
\begin{picture}(10,10)  
\put(260,-20){\mbox{\tiny{\today}}}
\end{picture}
}

\newcommand{\pd}[2]{\dfrac{\partial#1}{\partial#2}}
\newcommand{\pD}[2]{\dfrac{\partial^2#1}{\partial#2^2}}
\newcommand{\pdd}[3]{\dfrac{\partial^2#1}{\partial#2 \partial#3}}


\normalem %stops the ulem package making all the emphs into underlines...
 
 
 
 \newcommand{\refandrev}[2]{
 \begin{small}
  \hspace{6cm}
  \begin{minipage}[r]{8cm}
  Stewart,    Chapter #1   \\
  Review:  \parbox[t]{6cm}{#2}
\end{minipage}
\end{small}
}



\newcounter{heading}
%\setcounter{section}{1}
\setcounter{heading}{0}

\newcommand{\makeheading}[1]{\medskip\begin{large}\noindent\textbf{{#1}}\end{large}\smallskip}

%\newenvironment{head}[1]{\medskip\stepcounter{heading}\noindent\textbf{\hspace{0.2cm}{#1}.}}{}
\newcommand{\newhead}[1]{\medskip\stepcounter{heading}\noindent\textbf{\hspace{0.2cm}{#1}.}}


\newcommand{\pf}[1]{\noindent\textit{Proof.}\vspace*{#1 cm}}
\newcommand{\sol}[1]{\noindent\textit{Solution.}\vspace*{#1 cm}}
\newcommand{\further}[1]{\begin{small}\noindent\textit{Further reading: #1}\end{small}}
\newcommand{\exr}[1]{\begin{footnotesize}\noindent\textit{\textbf{Exercises:} Stewart #1}\end{footnotesize}}


% Sets of numbers
\newcommand{\C}{\mathbb{C}}
\newcommand{\RR}{\mathbb{R}}
\newcommand{\Z}{\mathbb{Z}}
\newcommand{\N}{\mathbb{N}}
\newcommand{\Q}{\mathbb{Q}}

% Partitions
\newcommand{\PP}{\mathcal{P}}

% Limits
\newcommand{\limm}[1]{\displaystyle \lim_{x\to #1}}

% Backslash
\newcommand{\bs}{\backslash}

% functions
\newcommand{\cosec}{\mathrm{cosec}}
\newcommand{\cosech}{\mathrm{cosech}}
\newcommand{\sech}{\mathrm{sech}}
\newcommand{\Li}{\mathrm{Li}}
\newcommand{\si}{\mathrm{Si}}
\newcommand{\erf}{\mathrm{erf}}

% Domain and Range
\newcommand{\Dom}{\mathrm{Dom}}
\newcommand{\Codom}{\mathrm{Codom}}
\newcommand{\Range}{\mathrm{Ran}}



\title{Week 13:  Final Exam Review Problems}
\date{October 31 -- November 1, 2012}

\begin{document}

\maketitle

\section{Limits and continuity}
\begin{enumerate}
\item Find $\limm{0}x^2\sin(1/x)$.

\item Find $\limm{\infty}\frac{\sin x}{x}$.

\item Show that the equation \[\sqrt[3]{x}=1-x \] has a solution in the interval $(0,1)$.

\item Evaluate $\limm{\infty}\big(\sqrt{x^2+x}-\sqrt{x^2-x}\big)$.

\item Find $\limm{\infty}\frac{x^2\sin x}{e^x}$.

\item Either evaluate
\[ \lim_{x\to\infty}\Big(e^x+1\Big)^{\tfrac{1}{x}} \]
or state that the limit does not exist. Give reasons for your answer.

\item Use l'Hospital's rule (or some other method) to find
\[ \limm{0}\dfrac{1-\cos x}{\ln(1+x^2)}. \]
\end{enumerate}

\section{Derivatives}
\begin{enumerate}
\item Differentiate $\tan^{-1}(x^4) + \cos^{-1}(e^x)$.

\item Differentiate $\ln (x^2 + e^{\cos x})$ and state for which values of $x$ the derivative is defined.

\item Find the domain of the function
\[ f (x) = \cos^{−1} (\ln x) + e^{\sin x} .\]
Find the derivative of $f$ and state for what values of $x$ it is defined.

\item Find the derivative of 
\[ F(x)\int_0^{x^2}e^{\sin t}\,dt. \]

\item Suppose that a function $f$ with domain $\mathbb{R}$ is defined by
\[f (x) = \left\lbrace\begin{array}{cl}
\dfrac{\sin x}{x} & \text{ if }x\neq 0\\
1 & \text{ if }x= 0
\end{array}\right.\]
\begin{enumerate}
\item Explain why $f$ is continuous on $\RR$.
\item Is $f$ differentiable at $0$? Give reasons for your answer.
\end{enumerate}

\medskip

\item The radius of a circle is measured to be $20 mm$ to the nearest millimetre. Using this
measurement, the area of the circle is then calculated to be $400\pi mm^2$ . Use the differential
approximation to estimate the maximum error for the calculated area of the circle.

\item Differentiate $\cosh^2(3x-1)$.

\item Compute the derivative of the function 
\[ A(t)=\int_1^{\cosh t}\sqrt{x^2-1}\dx \quad \text{ for }t>0. \] Name any theorems that you use.
\end{enumerate}

\section{Integration}

\begin{enumerate}
\item Evaluate $\ds \int_0^1\dfrac{e^x}{1+e^{2x}}\dx$.

\item Find $\ds\int_{-\pi/4}^{\pi/4}\dfrac{\sec^2x}{\sqrt{1-\tan^2x}}\dx$.

\item Find $\ds\int x(\ln x)^2\dx$.

\item Use the method of partial fractions to find
$\ds\int \dfrac{x+2}{x^2 +x}\dx$.

\item Suppose that \[ g(x)=\dfrac{1}{x^2+2x+5}.\]
Calculate the average value of $g$ over the interval $[1, 3]$.

\item Find \[ \int \tan^5x\sec^8x\dx. \]

\item Use the formula $ \sin A\cos B = \dfrac{1}{2}\big(\sin(A+B)+\sin(A-B)\big)$
to find \[ \int \sin(3x)\cos(x)\dx. \]

\item Evaluate $\ds\int_0^1\dfrac{\tan^{-1}x}{x^2+1}\dx$. 
\end{enumerate}

\section{Inverse functions}

\begin{enumerate}
\item Find the domain of the function $f(x)=\tan(\cos^{-1}(x))$. Can you simplify it?

\item Find the domain and range of the function
\[ f(x)=\ln(\cos x). \]

\item Find the domain and range of the function
\[ f(x)=\sin^{-1}\Big(\dfrac{x}{|x|}\Big). \]

\item If $\sinh x =\tfrac{3}{4}$, then find $\tanh x$.

\item Consider the function $f$ with domain $\RR$ defined by $f(x)=x^5+5x^3+10x$.
\begin{enumerate}
\item Is the function one-to-one? Justify your answer.
\item Compute the derivative $f^{-1}(t)$ at $t=0$.
\end{enumerate}

\end{enumerate}


\end{document}